\chapter{最优化算法}

\begin{problem}[无约束极小化]
	\begin{equation}
		\min_x f(x) 
	\end{equation}
\end{problem}

\section{停止准则}

\section{步长}

\subsection{Backtracking Search}



\section{直接搜索}
指无需对目标函数求导的搜索方法, 比如文献\cite{schutze2011directed}中的方法.

\subsection{单纯形法(Simplex Search Method)}


\section{梯度法(Gradient Method)}

\subsection{梯度法(Gradient Descent Method)}

梯度法或称梯度下降法.

\begin{algorithm}
\caption{梯度法}
\label{alg:gradient_descent}
{\bf Require:} $x_0\in {\bf R}^n$
\begin{algorithmic}[1]
\For{$k=0,1,\ldots$}
	\State $x_{k+1}=y_k-t_k\nabla f(y_k)$
\EndFor
\end{algorithmic}
\end{algorithm}

\subsection{加速邻近梯度法(Accelerated Proximal Gradient Method)}

此方法由Nesterov\cite{nesterov1983method}首先提出. 其中, $\theta_{k+1}\in (0,1)$见文献\cite{nesterov1998introductory} 91页. 当$q=1$时, 该算法为梯度法.

\begin{algorithm}
\caption{加速邻近梯度法\cite{nesterov1983method,nesterov1998introductory}(APG)}
\label{alg:accelerated_proximal_gradient}
{\bf Require:} $x_0\in {\bf R}^n, y_0=x_0, \theta_0=1, q\in[0,1]$
\begin{algorithmic}[1]
\For{$k=0,1,\ldots$}
	\State $x_{k+1}=y_k-t_k\nabla f(y_k)$
	\State $\theta_{k+1}^2=(1-\theta_{k+1})\theta_k^2+q\theta_{k+1}, \theta_{k+1}\in (0,1)$, 求解 $\theta_{k+1}$
	\State $\beta_{k+1}=\theta_k(1-\theta_k)/(\theta_k^2+\theta_{k+1})$
	\State $y_{k+1}=x_{k+1}+\beta_{k+1}(x_{k+1}-x_k)$
\EndFor
\end{algorithmic}
\end{algorithm}

\subsection{自适应重启加速梯度法}

加速邻近梯度法(算法\ref{alg:accelerated_proximal_gradient})迭代到一定程度时, 外推系数$\beta_k$趋于0, 算法退化成梯度法(算法\ref{alg:gradient_descent}). 使用该算法自适应的重置参数, 能够保持加速邻近梯度法的快速收敛.

\begin{algorithm}
\caption{自适应重启加速梯度法\cite{o2015adaptive}}
\label{alg:adaptive_restart_accelerated_proximal_gradient}
{\bf Require:} $x_0\in {\bf R}^n, y_0=x_0, \theta_0=1$
\begin{algorithmic}[1]
\For{$j=0,1,\ldots$}
	\State 取$q=0$执行算法\ref{alg:accelerated_proximal_gradient}, 直到 $f(x_k)>f(x_{k-1})$时停止 (或可以在 $\nabla f(y_{k-1})^T(x_k-x_{k-1})$时停止, 两者选其中一种)
	\State $x_0=x_k, y_0=x_k, \theta_0=1$
\EndFor
\end{algorithmic}
\end{algorithm}





\section{牛顿法及拟牛顿法}

\subsection{Broyden–Fletcher–Goldfarb–Shanno Algorithm(BFGS)}