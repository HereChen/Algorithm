\documentclass{book}

\usepackage{xeCJK}

% 算法
\usepackage[noend]{algpseudocode}
\usepackage{algorithmicx,algorithm}

\begin{document}

\title{算法简化描述}
\author{陈磊}
\date{\today}

\maketitle

\tableofcontents
\listofalgorithms

\chapter{非线性方程求根}

\chapter{线性方程组求解}

\chapter{微分方程数值解}

\section{龙格库塔(Runge Kutta)}
\subsection{四阶龙格库塔}
\begin{equation}
	\frac{dy}{dx}=f(x,y), y(0)=y_0
\end{equation}

\begin{algorithm}
\caption{四阶龙格库塔} %算法的名字
\hspace*{0.02in} {\bf Input:} %算法的输入, \hspace*{0.02in}用来控制位置,同时利用 \\ 进行换行
$y_0$, $h$, $n$\\
\hspace*{0.02in} {\bf Output:} %算法的结果输出
$y_i$
\begin{algorithmic}[1]
\For{i=1:n}
	\State $k_1=f(x_i,y_i)$
	\State $k_2=f(x_i+\frac{1}{2}h,y_i+\frac{1}{2}k_1h)$
	\State $k_3=f(x_i+\frac{1}{2}h,y_i+\frac{1}{2}k_2h)$
	\State $k_4=f(x_i+h,y_i+k_3h)$
	\State $y_{i+1} = y_i + \frac{1}{6}(k_1+2k_2+2k_3+k_4)h$
\EndFor
\end{algorithmic}
\end{algorithm}

求解微分方程组.

\chapter{最优化}

\chapter{随机搜索}
\section{遗传算法(Genetic Algorithm, GA)}
\section{粒子群算法(Particle Swarm Optimization, PSO)}

\chapter{数据分析}
\section{聚类}

\end{document}