\documentclass[a4paper,11pt,oneside]{book}

\usepackage{xeCJK}

% 生成PDF书签
\usepackage{hyperref}

% 算法
\usepackage[noend]{algpseudocode}
\usepackage{algorithmicx,algorithm}

\begin{document}

\title{算法简化描述}
\author{陈磊}
\date{\today}

\maketitle
\frontmatter

\tableofcontents
\listofalgorithms
% \listoffigures
% \listoftables

\mainmatter

\chapter{非线性方程求根}

\chapter{线性方程组求解}

\chapter{微分方程数值解}

\section{龙格库塔(Runge Kutta)}
\subsection{四阶龙格库塔}
\begin{equation}
	\frac{dy}{dx}=f(x,y), y(0)=y_0
\end{equation}

\begin{algorithm}
\caption{四阶龙格库塔} %算法的名字
{\bf Require:} $f, y_0, h, n$
\begin{algorithmic}[1]
\For{$i=0,1,\ldots$}
	\State $k_1=f(x_i,y_i)$
	\State $k_2=f(x_i+\frac{1}{2}h,y_i+\frac{1}{2}k_1h)$
	\State $k_3=f(x_i+\frac{1}{2}h,y_i+\frac{1}{2}k_2h)$
	\State $k_4=f(x_i+h,y_i+k_3h)$
	\State $y_{i+1} = y_i + \frac{1}{6}(k_1+2k_2+2k_3+k_4)h$
\EndFor
\end{algorithmic}
\end{algorithm}

求解微分方程组.

\chapter{最优化算法}
\section{梯度法(Gradient Method)}

\subsection{梯度法(Gradient Descent Method)}

梯度法或称梯度下降法.

\begin{algorithm}
\caption{梯度法}
\label{alg:APG}
{\bf Require:} $x_0\in {\bf R}^n$
\begin{algorithmic}[1]
\For{$k=0,1,\ldots$}
	\State $x_{k+1}=y_k-t_k\nabla f(y_k)$
\EndFor
\end{algorithmic}
\end{algorithm}

\subsection{加速邻近梯度法(Accelerated Proximal Gradient Method, APG)}

此方法由Nesterov\cite{nesterov1983method}首先提出.

\begin{algorithm}
\caption{加速邻近梯度法\cite{nesterov1983method,nesterov1998introductory}}
\label{alg:APG}
{\bf Require:} $x_0\in {\bf R}^n, y_0=x_0, \theta_0=1, q\in[0,1]$
\begin{algorithmic}[1]
\For{$k=0,1,\ldots$}
	\State $x_{k+1}=y_k-t_k\nabla f(y_k)$
	\State $\theta_{k+1}^2=(1-\theta_{k+1})\theta_k^2+q\theta_{k+1}, \theta_{k+1}\in (0,1)$, 求解 $\theta_{k+1}$
	\State $\beta_{k+1}=\theta_k(1-\theta_k)/(\theta_k^2+\theta_{k+1})$
	\State $y_{k+1}=x_{k+1}+\beta_{k+1}(x_{k+1}-x_k)$
\EndFor
\end{algorithmic}
\end{algorithm}
其中, $\theta_{k+1}\in (0,1)$见文献\cite{nesterov1998introductory} 91页. 当$q=1$时, 该算法为梯度法.

\subsection{自适应重启加速梯度法}

加速邻近梯度法(算法\ref{alg:APG})迭代到一定程度时, 外推系数$\beta_k$趋于0, 算法退化成梯度法. 使用该算法自适应的重置参数, 能够保持加速邻近梯度法的快速收敛.

\begin{algorithm}
\caption{自适应重启加速梯度法\cite{o2015adaptive}}
{\bf Require:} $x_0\in {\bf R}^n, y_0=x_0, \theta_0=1$
\begin{algorithmic}[1]
\For{$j=0,1,\ldots$}
	\State 取$q=0$执行算法\ref{alg:APG}, 直到 $f(x_k)>f(x_{k+1})$时停止 (或可以在 $\nabla f(y_{k-1})^T(x_k-x_{k-1})$时停止, 两者选其中一种)
	\State $x_0=x_k, y_0=x_k, \theta_0=1$
\EndFor
\end{algorithmic}
\end{algorithm}

\chapter{随机搜索}
\section{遗传算法(Genetic Algorithm, GA)}
\section{粒子群算法(Particle Swarm Optimization, PSO)}

\chapter{数据分析}
\section{聚类}

\backmatter
\bibliographystyle{plain}
\bibliography{bibliography}

\end{document}